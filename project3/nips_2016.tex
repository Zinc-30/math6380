\documentclass{article}
\usepackage{listings}
% if you need to pass options to natbib, use, e.g.:
% \PassOptionsToPackage{numbers, compress}{natbib}
% before loading nips_2016
%
% to avoid loading the natbib package, add option nonatbib:
% \usepackage[nonatbib]{nips_2016}

\usepackage[final]{nips_2016}

% to compile a camera-ready version, add the [final] option, e.g.:
% \usepackage[final]{nips_2016}

\usepackage[utf8]{inputenc} % allow utf-8 input
\usepackage[T1]{fontenc}    % use 8-bit T1 fonts
\usepackage{hyperref}       % hyperlinks
\usepackage{url}            % simple URL typesetting
\usepackage{booktabs}       % professional-quality tables
\usepackage{amsfonts}       % blackboard math symbols
\usepackage{nicefrac}       % compact symbols for 1/2, etc.
\usepackage{microtype}      % microtypography

\title{mini project2}

% The \author macro works with any number of authors. There are two
% commands used to separate the names and addresses of multiple
% authors: \And and \AND.
%
% Using \And between authors leaves it to LaTeX to determine where to
% break the lines. Using \AND forces a line break at that point. So,
% if LaTeX puts 3 of 4 authors names on the first line, and the last
% on the second line, try using \AND instead of \And before the third
% author name.

\author{
  XIN Hao \\
  Jian Xun \\
  Yu Jinxing \\
  Department of Computer Science\\
}

\begin{document}
% \nipsfinalcopy is no longer used

\maketitle
%!TEX root = nips_2016.tex
\begin{abstract}
This is a short report for the mini-project2 of the class Math6380 in HKUST. The project we choose is a regression problem.
\end{abstract}

%!TEX root = nips_2016.tex
\section{introduction}
We choosed Drug Sensitivity Ranking and participated in the kaggle inclass contest.  We are the team "3654". Our best submission gets accuracy of \textbf{94.74\%}
The dataset provides 265 Drugs with experiments on 990 cancer cell lines. Drug sensitivity is measured by IC50. There are 3 million comparosion for defferent durgs on each cells, and the result can be 3 types. For example, in cell1, the drug1 may be more sensitivity than drug2, then the pair <drug1,drug2,cell1> will be 1. Among the 3M samples, 1/5 (.6M) comparison values are missing. Our job is to predict these values based on the information given. \\ 
In particular, we also have the gene describtion of eath cell line, which is a binary genetic features of dimensionality 1250.




\section{Methodology}
%!TEX root = nips_2016.tex
\subsection{data preprocessing}
We use both given files. The dataset is clean. We just need to build the training dataset and testing dataset from it. \\


\subsection{feature engeneering}
The feature dimension of this dataset is not quite high. And they all have actual meanings, presenting dosage levels of 20 different drugs. So we didn't manually create other features using their combinations.
% To cope with the high-dimension of features and prevent the learning algorithms from overfitting, we performed dimension reduction or feature selection on features.  We compared severial regression methods including LASSO, Ridge Regression, Support Vector Regression, Decision Tree, and Gradient Boosting.  Two python packages:scikit-learn and pandas were used in the implementation. \\
% We also manually implemented a recursive feature selection method from scratch based on Lasso regression errors on cross validation data. Different from Lasso recursive feature elimination in the reference poster which repeatly eliminates some features, our feature selection method repeatly add a feature until the Lasso regression error on validation dataset does not decrease. The advantage of our method is that it is more efficient than recursive feature elimination. 

\subsection{model selection}
In this part, We did 5-fold cross validation on the training dataset for model selection. Our models includes LASSO, Ridge Regression, Support Vector Regression, Elastic Net and Gradient Boosting. \\
We think the key point of this project is how to tune the parameters.

\section{Results and Discussion}
We first compare different regression method based on the Mean Square Error through 5-fold cross validation. The training errors are also included to check the overfitting. We then picked several models and trained them on the whole training dataset and submitted them on Kaggle.
\begin{table}[htbp]
\centering
\begin{tabular}{|l|l|l|}
\hline
model & validation error & training error \\
\hline
lasso($\alpha=0.005$)         &  0.013677 & 0.008775 \\
\hline
Elastic Net($\alpha=0.01,ratio_{l_1}=0.5$)   &  0.013680 & 0.008796 \\
\hline
ridge($\alpha=20$)         &  0.014146 & 0.009051 \\
\hline
bayes ridge($\alpha_1=10,\alpha_2=0.010$)  &  0.014197 & 0.008665 \\
\hline
svr        &  0.019456 & 0.006158 \\
\hline
lars       & 0.021611 & 0.020714 \\
\hline
kernel ridge($\alpha=0.01$)  &  0.021955 & 0.014500 \\
\hline
lasso Lars($\alpha=0.1$)    &  0.022331 & 0.021887 \\
\hline
boost         &  0.023385 & 0.000494 \\
\hline
\end{tabular}
\caption{Cross validation results measured by mean square error for each method}
\end{table}
From the results in Table1, 
We found is that the difference between these regression models' validation error is not quite big. And based on the result we get in the kaggle testing, all team get training error lower than 0.014. We think this problem is a simple task.

\section{Remark on Contributions}
The project is finished under the discussion and close collaboration of our group members. Hao Xin wrote the code skeleton and wrote the majority of the report draft. Jinxing Yu tuned many regression methods and their parameters by cross validation and tried to get a better test result. Xun Jian proposed some feature selection method and implemented it, which got the best cross validation result.

\section*{Appendix}
\begin{lstlisting}[language=c++]
#include <iostream>
#include <math.h>
#include <string>
#include <map>
#include <fstream>
#include <sstream>
#include <vector>
#include <algorithm>
#include <random>

using namespace std;
#define MAXSTRING 100
#define CELL_SIZE 990
map<string, int> drug_dict;
map<string, int> cell_dict;
typedef double real;
real lr = 1e-4;
real weight_scale = 1e-3;
int batch_size = 1000;
int epoch = 1000;
int drug_size = 265, gene_size = 1250, class_size = 3;
vector<int> cell_feature[CELL_SIZE];
int train_data_nums = 0;
int test_data_nums = 0;
int print_epo = 1;
/*
 Model parameters:
 drugW, the embedding matrix of drugs
 drugb, the bias vector of drugs
 affineW, affineb, the affine layer, convert the rank scores to softmax scores
 cell_embed, temp vector to store the embedding of cell
 */
real *drugW, *drugb, *affineW, *affineb;
// gradients of parameters
real *grad_drugW, *grad_drugb, *grad_affineW, *grad_affineb;
struct DataPair{
    int sample_id;
    int cell_id;
    int drug1_id;
    int drug2_id;
    int y;
};
vector<DataPair> train_pairs;
vector<DataPair> test_pairs;
void ReadDrugDict(char* filename)
{
    ifstream infile(filename);
    if (infile.is_open()){
        string tmp_drug;
        int idx = 0;
        while (infile >> tmp_drug){
            drug_dict[tmp_drug] = idx;
            idx++;
        }
        infile.close();
    }
    else{
        printf("ERROR: failed to read the file in ReadDrugDict\n");
        exit(1);
    }
}
void ReadCellDict(char* filename)
{
    ifstream infile(filename);
    if (infile.is_open()){
        string tmp_cell;
        int idx = 0;
        while (infile >> tmp_cell){
            cell_dict[tmp_cell] = idx;
            idx++;
        }
        infile.close();
    }
    else{
        printf("ERROR: failed to read the file in ReadDrugDict\n");
        exit(1);
    }
}
// Read the features of the drug
void ReadCellFeature(char* filename)
{
    ifstream infile(filename);
    if (infile.is_open()){
        int cnt = 0;
        string line;
        while(getline(infile,line)){
            istringstream iss(line);
            int tmp;
            while(iss >> tmp)
                cell_feature[cnt].push_back(tmp);
            cnt++;
        }
        infile.close();
    }
    else{
        printf("ERROR: failed to read the file of cell features\n");
        exit(1);
    }
}
// Load the training pairs
void ReadTrainPairs(char* filename)
{
    ifstream infile(filename);
    if (infile.is_open()){
        string line;
        while (getline(infile, line)){
            istringstream iss(line);
            DataPair tmp;
            iss >> tmp.sample_id;
            tmp.sample_id += 1;
            string cell_name, drug1_name, drug2_name;
            iss >> cell_name;
            iss >> drug1_name;
            iss >> drug2_name;
            tmp.cell_id = cell_dict[cell_name];
            tmp.drug1_id = drug_dict[drug1_name];
            tmp.drug2_id = drug_dict[drug2_name];
            float y = 0;
            iss >> y;
            tmp.y = int(y);
            train_pairs.push_back(tmp);
        }
        infile.close();
        train_data_nums = (int) train_pairs.size();
    }
    else{
        printf("ERROR: failed to read the file of training pairs\n");
        exit(1);
    }
}
// Load the testing pairs
void ReadTestPairs(char* filename)
{
    ifstream infile(filename);
    if (infile.is_open()){
        string line;
        while (getline(infile, line)){
            istringstream iss(line);
            DataPair tmp;
            tmp.y = 0;  // intialize the unknow value to 0
            iss >> tmp.sample_id;
            tmp.sample_id += 1;
            string cell_name, drug1_name, drug2_name;
            iss >> cell_name;
            iss >> drug1_name;
            iss >> drug2_name;
            tmp.cell_id = cell_dict[cell_name];
            tmp.drug1_id = drug_dict[drug1_name];
            tmp.drug2_id = drug_dict[drug2_name];
            test_pairs.push_back(tmp);
        }
        infile.close();
        test_data_nums = int(test_pairs.size());
    }
    else{
        printf("ERROR: failed to read the file of testing pairs\n");
        exit(1);
    }
}

void ClearGradients()
{
    for (int i = 0; i < drug_size * gene_size; i++) grad_drugW[i] = 0;
    for (int i = 0; i < drug_size; i++) grad_drugb[i] = 0;
    for (int i = 0; i < class_size; i++) grad_affineW[i] = 0;
    for (int i = 0; i < class_size; i++) grad_affineb[i] = 0;
}

void UpdateParameters()
{
    for (int i = 0; i < drug_size * gene_size; i++) drugW[i] += - lr * grad_drugW[i];
    for (int i = 0; i < drug_size; i++) drugb[i] += -lr * grad_drugb[i];
    for (int i = 0; i < class_size; i++) affineW[i] += -lr * grad_affineW[i];
    for (int i = 0; i < class_size; i++) affineb[i] += -lr * grad_affineb[i];
}

struct Result{
    int pred_y;
    real loss;
};

// Predict the drug sensitivity rank for each data sample
// Only backpropagate the gradients during the training period
Result TrainPredictOneData(DataPair cur_pair, bool train)
{
    Result result;
    // forward pass
    int tmp_cell_id = cur_pair.cell_id;

    int tmp_gene_num = (int) cell_feature[tmp_cell_id].size();
    int tmp_drug1_id = cur_pair.drug1_id;
    int tmp_drug2_id = cur_pair.drug2_id;
    // computhe the score s = b1 - b2 + <x, beta_1 - beta_2>
    real s = drugb[tmp_drug1_id] - drugb[tmp_drug2_id];
    for (int k = 0; k < tmp_gene_num; k++){
        int gene_id = cell_feature[tmp_cell_id][k];
        s += drugW[tmp_drug1_id * gene_size + gene_id] - drugW[tmp_drug2_id * gene_size + gene_id];
    }
    
    // transform the score s to target labels probability by affine - softmax layers
    real prob[3];
    real maxp = 0;
    int maxid = 0;
    for (int k = 0; k < 3; k++){
        prob[k] = s * affineW[k] + affineb[k];
        if (prob[k] > maxp){
            maxp = prob[k];
            maxid = k;
        }
    }
    
    // transform by softmax function
    real sum = 0;
    for (int k = 0; k < 3; k++){
        prob[k] -= maxp;
        prob[k] = exp(prob[k]);
        sum += prob[k];
    }
    
    for (int k = 0; k < 3; k++)
        prob[k] /= sum;
    result.pred_y = maxid;
    result.loss = 0;
    if (train){
        int tmp_y = cur_pair.y;
        result.loss = -log(prob[tmp_y+1]);
        // backpropagate the gradients
        real dx[3];
        for (int k = 0; k < 3; k++)
            dx[k] = prob[k];
        dx[tmp_y+1] -= 1;
        for (int k = 0; k < 3; k++){
            grad_affineW[k] += dx[k] * s;
            grad_affineb[k] += dx[k];
        }
        real ds = 0;
        for (int k = 0; k < 3; k++)
            ds += dx[k] * affineW[k];

        grad_drugb[tmp_drug1_id] += ds;
        grad_drugb[tmp_drug2_id] += -ds;
        for (int k = 0; k < tmp_gene_num; k++){
            int gene_id = cell_feature[tmp_cell_id][k];
            grad_drugW[tmp_drug1_id * gene_size + gene_id] += ds;
            grad_drugW[tmp_drug2_id * gene_size + gene_id] += -ds;
        }
    }
    return result;
}
// partition the training data into training set and validation set
void Train()
{
    // random shuffle the training data
    random_shuffle(train_pairs.begin(), train_pairs.end());
    int sample_train_size = train_data_nums / 5 * 4;
    real loss = 0, train_acc = 0, val_acc = 0;
    for (int iter = 0; iter < epoch; iter++){
        // train and update parameters
        for (int i = 0; i < sample_train_size; i = i + batch_size){
            real batch_loss = 0;
            ClearGradients();
            for (int j = i; j < i + batch_size && j < sample_train_size; j++){
                Result result = TrainPredictOneData(train_pairs[j], true);
                batch_loss += result.loss;
            }
            // printf("batch loss %.4f\n",batch_loss);
            // update parameters by SGD
            UpdateParameters();
            
            loss += batch_loss;
        }
        loss /= sample_train_size / batch_size;
        // train accuracy
        for (int i = 0; i < sample_train_size; i++){
          Result result = TrainPredictOneData(train_pairs[i], false);
          if (result.pred_y == train_pairs[i].y + 1)
            train_acc += 1.0 / sample_train_size;
        }
        // validation accuracy
        for (int i = sample_train_size; i < train_data_nums; i++){
            Result result = TrainPredictOneData(train_pairs[i], false);
            if (result.pred_y == train_pairs[i].y + 1)
                val_acc += 1.0 / (train_data_nums - sample_train_size);
        }
        // print average loos and training accuracy
        if ((iter % print_epo) == 0){
            printf("==========> epoch %d  loss %.4f train-acc %.4f val-acc %.4f\n", iter, loss / print_epo, train_acc / print_epo, val_acc / print_epo);
            loss = 0;
            train_acc = 0;
            val_acc = 0;
        }
    }
}
// use the whole dataset for training
void Train2()
{
    // random shuffle the training data
    random_shuffle(train_pairs.begin(), train_pairs.end());
    real loss = 0, train_acc = 0;
    for (int iter = 0; iter < epoch; iter++){
        // train and update parameters
        for (int i = 0; i < train_data_nums; i = i + batch_size){
            real batch_loss = 0;
            ClearGradients();
            for (int j = i; j < i + batch_size && j < train_data_nums; j++){
                Result result = TrainPredictOneData(train_pairs[j], true);
                batch_loss += result.loss;
            }
            // printf("batch loss %.4f\n",batch_loss);
            // update parameters by SGD
            UpdateParameters();
            
            loss += batch_loss;
        }
        loss /= train_data_nums / batch_size;
        // train accuracy
        for (int i = 0; i < train_data_nums; i++){
            Result result = TrainPredictOneData(train_pairs[i], false);
            if (result.pred_y == train_pairs[i].y + 1)
                train_acc += 1.0 / train_data_nums;
        }
        // print average loos and training accuracy
        if ((iter % print_epo) == 0){
            printf("==========> epoch %d  loss %.4f train-acc %.4f\n", iter, loss / print_epo, train_acc / print_epo);
            loss = 0;
            train_acc = 0;
        }
    }
}
void Test(char *test_predict_file)
{
    // predict the drug pairs in test file
    ofstream ofile(test_predict_file);
    ofile << "SampleID,ComparisonValue" << endl;
    for (int i = 0; i < test_data_nums; i++){
        Result result = TrainPredictOneData(test_pairs[i], false);
        test_pairs[i].y = result.pred_y;
        ofile << test_pairs[i].sample_id << "," << test_pairs[i].y - 1 << endl;
    }
    ofile.close();
}
void normal_initialize(real *array, int asize)
{
    unsigned seed = std::chrono::system_clock::now().time_since_epoch().count();
    std::default_random_engine generator (seed);
    std::normal_distribution<double> distribution (0.0,1.0);
    for (int i = 0; i < asize; i++)
        array[i] = weight_scale * distribution(generator);
}
void SaveParameters(char* parameter_file)
{
    ofstream ofile(parameter_file);
    ofile << "drugW" << endl;
    ofile << drugW[0];
    int i = 0;
    for (i = 1; i < drug_size * gene_size; i++)
        ofile << " " << drugW[i];
    ofile << endl;
    ofile << "drugb" << endl;
    ofile << drugb[0];
    for (i = 1; i < drug_size; i++)
        ofile << " " << drugb[i];
    ofile << endl;
    ofile << "affineW" << endl;
    ofile << affineW[0];
    for (i = 1; i < class_size; i++)
        ofile << " " << affineW[i];
    ofile << endl;
    ofile << "affineb" << endl;
    ofile << affineb[0];
    for (i = 1; i < class_size; i++)
        ofile << " " << affineb[i];
    ofile << endl;
    ofile.close();
}
void ReadParameters(char *parameter_file)
{
    ifstream infile(parameter_file);
    if (infile.is_open()){
        string name;
        string line;
        int i = 0;
        getline(infile,name);
        getline(infile,line);
        istringstream iss(line);
        for (i = 0; i < drug_size * gene_size; i++)
            iss >> drugW[i];
        getline(infile,name);
        getline(infile,line);
        iss.str(line);
        for (i = 0; i < drug_size; i++)
            iss >> drugb[i];
        getline(infile,name);
        getline(infile,line);
        iss.str(line);
        for (i = 0; i < class_size; i++)
            iss >> affineW[i];
        getline(infile,name);
        getline(infile,line);
        iss.str(line);
        for (i = 0; i < class_size; i++)
            iss >> affineb[i];
        infile.close();
    }
    else{
        printf("ERROR: failed to read the parameter file\n");
        exit(1);
    }
}
void InitParameters()
{
    drugW = (real *)calloc(drug_size * gene_size, sizeof(real));
    drugb = (real *)calloc(drug_size, sizeof(real));
    affineW = (real *)calloc(class_size, sizeof(real));
    affineb = (real *)calloc(class_size, sizeof(real));

    normal_initialize(drugW, drug_size * gene_size);
    normal_initialize(affineW, class_size);

    grad_drugW = (real *)calloc(drug_size * gene_size, sizeof(real));
    grad_drugb = (real *)calloc(drug_size, sizeof(real));
    grad_affineW = (real *)calloc(class_size, sizeof(real));
    grad_affineb = (real *)calloc(class_size, sizeof(real));
}
void DestroyParameters()
{
    free(drugW);
    free(drugb);
    free(affineW);
    free(affineb);

    free(grad_drugW);
    free(grad_drugb);
    free(grad_affineW);
    free(grad_affineb);
}

int ArgPos(char *str, int argc, char **argv)
{
    int a;
    for (a = 1; a < argc; a++) if (!strcmp(str, argv[a])) {
        if (a == argc - 1) {
            printf("Argument missing for %s\n", str);
            exit(1);
        }
        return a;
    }
    return -1;
}

int main(int argc, char**argv)
{
    char drug_list_file[MAXSTRING], cell_list_file[MAXSTRING], cell_feature_file[MAXSTRING];
    char train_pairs_file[MAXSTRING], test_pairs_file[MAXSTRING], test_predict_file[MAXSTRING];
    
    char parameter_file[MAXSTRING] = "parameters.txt";
    int i = 0;
    if (argc == 1) {
        printf("Drug sensitivity rank prediction \n\n");
        printf("Options:\n");
        printf("Parameters for training:\n");
        printf("\t-druglist <file>\n");
        printf("\t\tUse <file> to build a dictionary of drug names\n");
        printf("\t-celllist <file>\n");
        printf("\t\tUse <file> to build a dictionary of cell names\n");
        printf("\t-cellfeature <file>\n");
        printf("\t\tUse <file> to get gene features for each cell\n");
        printf("\t-trainpairs <file>\n");
        printf("\t\tUse <file> to get the training pairs\n");
        printf("\t-testpairs <file>\n");
        printf("\t\tUse <file> to get the testing pairs\n");
        printf("\t-testpredict <file>\n");
        printf("\t\tUse <file> to store the predict results\n");
        printf("\nExamples:\n");
        printf("./rank -druglist druglist.txt -celllist celllist.txt -cellfeature cellfeature.txt -trainpairs trainpair.txt -testpairs testpair.txt -testpredict test_predict.csv \n\n");
        return 0;
    }
    
    drug_list_file[0] = 0;
    cell_list_file[0] = 0;
    cell_feature_file[0] = 0;
    train_pairs_file[0] = 0;
    test_pairs_file[0] = 0;
    test_predict_file[0] = 0;
               
    if ((i = ArgPos((char *)"-druglist", argc, argv)) > 0)    strcpy(drug_list_file, argv[i + 1]);
    if ((i = ArgPos((char *)"-celllist", argc, argv)) > 0)    strcpy(cell_list_file, argv[i + 1]);
    if ((i = ArgPos((char *)"-cellfeature", argc, argv)) > 0) strcpy(cell_feature_file, argv[i + 1]);
    if ((i = ArgPos((char *)"-trainpairs", argc, argv)) > 0) strcpy(train_pairs_file, argv[i + 1]);
    if ((i = ArgPos((char *)"-testpairs", argc, argv)) > 0) strcpy(test_pairs_file, argv[i + 1]);
    if ((i = ArgPos((char *)"-testpairs", argc, argv)) > 0) strcpy(test_pairs_file, argv[i + 1]);
    if ((i = ArgPos((char *)"-testpredict", argc, argv)) > 0) strcpy(test_predict_file, argv[i + 1]);
               
    ReadDrugDict(drug_list_file);
    ReadCellDict(cell_list_file);
    ReadCellFeature(cell_feature_file);
    ReadTrainPairs(train_pairs_file);
    ReadTestPairs(test_pairs_file);
    InitParameters();
    Train2();
    Test(test_predict_file);
    SaveParameters(parameter_file);
    DestroyParameters();
    return 0;
}
\end{lstlisting}

% \subsubsection*{Acknowledgments}

% Use unnumbered third level headings for the acknowledgments. All
% acknowledgments go at the end of the paper. Do not include
% acknowledgments in the anonymized submission, only in the final paper.

% \section*{References}

% References follow the acknowledgments. Use unnumbered first-level
% heading for the references. Any choice of citation style is acceptable
% as long as you are consistent. It is permissible to reduce the font
% size to \verb+small+ (9 point) when listing the references. {\bf
%   Remember that you can use a ninth page as long as it contains
%   \emph{only} cited references.}
% \medskip

% \small

% [1] Alexander, J.A.\ \& Mozer, M.C.\ (1995) Template-based algorithms
% for connectionist rule extraction. In G.\ Tesauro, D.S.\ Touretzky and
% T.K.\ Leen (eds.), {\it Advances in Neural Information Processing
%   Systems 7}, pp.\ 609--616. Cambridge, MA: MIT Press.

% [2] Bower, J.M.\ \& Beeman, D.\ (1995) {\it The Book of GENESIS:
%   Exploring Realistic Neural Models with the GEneral NEural SImulation
%   System.}  New York: TELOS/Springer--Verlag.

% [3] Hasselmo, M.E., Schnell, E.\ \& Barkai, E.\ (1995) Dynamics of
% learning and recall at excitatory recurrent synapses and cholinergic
% modulation in rat hippocampal region CA3. {\it Journal of
%   Neuroscience} {\bf 15}(7):5249-5262.

\end{document}
