\documentclass{article}
\usepackage{listings}
% if you need to pass options to natbib, use, e.g.:
% \PassOptionsToPackage{numbers, compress}{natbib}
% before loading nips_2016
%
% to avoid loading the natbib package, add option nonatbib:
% \usepackage[nonatbib]{nips_2016}

\usepackage[final]{nips_2016}

% to compile a camera-ready version, add the [final] option, e.g.:
% \usepackage[final]{nips_2016}

\usepackage[utf8]{inputenc} % allow utf-8 input
\usepackage[T1]{fontenc}    % use 8-bit T1 fonts
\usepackage{hyperref}       % hyperlinks
\usepackage{url}            % simple URL typesetting
\usepackage{booktabs}       % professional-quality tables
\usepackage{amsfonts}       % blackboard math symbols
\usepackage{nicefrac}       % compact symbols for 1/2, etc.
\usepackage{microtype}      % microtypography
\usepackage{amsmath}
\title{mini project3}

% The \author macro works with any number of authors. There are two
% commands used to separate the names and addresses of multiple
% authors: \And and \AND.
%
% Using \And between authors leaves it to LaTeX to determine where to
% break the lines. Using \AND forces a line break at that point. So,
% if LaTeX puts 3 of 4 authors names on the first line, and the last
% on the second line, try using \AND instead of \And before the third
% author name.

\author{
  Xin Hao \quad Jian Xun \quad Yu Jinxing \\
  Department of Computer Science\\
}

\begin{document}
% \nipsfinalcopy is no longer used

\maketitle
%!TEX root = nips_2016.tex
\begin{abstract}
This is a short report for the mini-project2 of the class Math6380 in HKUST. The project we choose is a regression problem.
\end{abstract}

%!TEX root = nips_2016.tex
\section{introduction}
We choosed Drug Sensitivity Ranking and participated in the kaggle inclass contest.  We are the team "3654". Our best submission gets accuracy of \textbf{94.74\%}
The dataset provides 265 Drugs with experiments on 990 cancer cell lines. Drug sensitivity is measured by IC50. There are 3 million comparosion for defferent durgs on each cells, and the result can be 3 types. For example, in cell1, the drug1 may be more sensitivity than drug2, then the pair <drug1,drug2,cell1> will be 1. Among the 3M samples, 1/5 (.6M) comparison values are missing. Our job is to predict these values based on the information given. \\ 
In particular, we also have the gene describtion of eath cell line, which is a binary genetic features of dimensionality 1250.



\section{Drug Sensitivity Rank Models}
In this section, we introduce our drug sensitivity rank models based on the standard HodgeRank score. Different from the vanilla HodgeRank score, our model further incorporates the binary features of celllines and uses a softmax function to predict the probabilities of different pairwise comparison values. We formulate the parwise rank sensitivity prediction as a classification problem and use the cross-entropy between the predict probability and the ground truth pairwise rank label as the training objective function. The objective function is optimized via SGD (Stochastic gradient decent). \\
Let there be M genes, N celllines, and K drugs. Let $X(k) \in \mathcal{R}^M$ be the binary genetic feature vector of cell line k, $\beta_1(i) \in \mathcal{R}^M$ be the vector representation of drug i, $\beta_0(i) \in \mathcal{R}$ be the inital score of drug i, where $\beta_1 \in \mathcal{R}^{K \times M}$ and  $\beta_0 \in \mathcal{R} ^{K}$ are  model parameters.
\\ 
Given a cell line k with genetic feature vector $X(k)$ and pairwise sensitivity of drug i and j on k, $y(k,i,j) \in \{-1, 0, 1\}$, the rank score is computed as 
\begin{equation}
s = \beta_0(i) - \beta_0(j) + X(K)^T (\beta_1(i) - \beta_1(j)).
\end{equation}
The score is transformed to prediction probability by 
\begin{equation}
p = softmax(Ws + b),
\end{equation}
where $p \in \mathcal{R}^3$, $p(0), p(1), p(2)$ indicate the probability that the parwise rank value is -1,0, or 1 resprectively, $W,b \in \mathcal{R}^3$ are model parameters, $softmax(x)_j = \frac{e^{x_j}}{\sum_{i=1}^K x_i}$.The overall loss is the negative log-likelihoods of  the rank prediction:
\begin{equation}
\mathcal{L} = - \log (p (y(k,i,j) + 1))
\end{equation}
Let $t = Ws + b$, $\hat{y}(k,i,j)\in \mathcal{R}^3$ be the one-hot encoding representation of ground truth label $y(k,i,j)$. The gradients of model parameters can be caculated by the chain rule:
\begin{align*}
\frac{\partial \mathcal{L}}{\partial t} &= p - \hat{y}(k,i,j) \\
\frac{\partial \mathcal{L}}{\partial W} &= \frac{\partial \mathcal{L}}{\partial t} s\\
\frac{\partial \mathcal{L}}{\partial b} &= \frac{\partial \mathcal{L}}{\partial t} \\
\frac{\partial \mathcal{L}}{\partial s} &= W^T \frac{\partial \mathcal{L}}{\partial t} \\
\frac{\partial \mathcal{L}}{\partial \beta_0(i)} &= \frac{\partial \mathcal{L}}{\partial s} \\
\frac{\partial \mathcal{L}}{\partial \beta_0(j)} &= -\frac{\partial \mathcal{L}}{\partial s} \\
\frac{\partial \mathcal{L}}{\partial \beta_1(i)} &=  X(k) \frac{\partial \mathcal{L}}{\partial s} \\
\frac{\partial \mathcal{L}}{\partial \beta_1(j)} &=  -X(k) \frac{\partial \mathcal{L}}{\partial s}
\end{align*}
Since the cell line feature $X_(k)$ is a binary feature vector, we also propose an extention to our model by adding model parameters $w_1 \in \mathcal{R}^ M$ to weight the cell line features. Then the rank score is modified as 
\begin{equation}
s = \beta_0(i) - \beta_0(j) + (X(K) \odot w_1)^T (\beta_1(i) - \beta_1(j)),
\end{equation}
where $\odot$ represents element-wise product of vectors.
\section{Methodology}
We tested both the original model (\textit{Original}) and the model with extension of $w_1$ (\textit{Extended}). For both models we used the same training settings, with epoch=1000, batch size=1000, learning rate=0.001, and initial weights following Gaussian distribution.
%%!TEX root = nips_2016.tex
\subsection{data preprocessing}
We use both given files. The dataset is clean. We just need to build the training dataset and testing dataset from it. \\

%\subsection{feature engeneering}
%The feature dimension of this dataset is not quite high. And they all have actual meanings, presenting dosage levels of 20 different drugs. So we didn't manually create other features using their combinations.
% To cope with the high-dimension of features and prevent the learning algorithms from overfitting, we performed dimension reduction or feature selection on features.  We compared severial regression methods including LASSO, Ridge Regression, Support Vector Regression, Decision Tree, and Gradient Boosting.  Two python packages:scikit-learn and pandas were used in the implementation. \\
% We also manually implemented a recursive feature selection method from scratch based on Lasso regression errors on cross validation data. Different from Lasso recursive feature elimination in the reference poster which repeatly eliminates some features, our feature selection method repeatly add a feature until the Lasso regression error on validation dataset does not decrease. The advantage of our method is that it is more efficient than recursive feature elimination. 

% \subsection{model selection}
% In this part, We did 5-fold cross validation on the training dataset for model selection. Our models includes LASSO, Ridge Regression, Support Vector Regression, Elastic Net and Gradient Boosting. \\
% We think the key point of this project is how to tune the parameters.

%!TEX root = nips_2016.tex
\section{Results and Discussion}
We compared two models in four aspects: training loss, training accuracy, validation accuracy, and test score from kaggle. The results are shown in Table 1. \\
\begin{table}[htbp]
\label{result}
\centering
\begin{tabular}{|l|l|l|l|l|}
\hline
model    & training loss & training accuracy & validation accuracy & test score \\
\hline
Original & 59.0390       & 0.9753            & 0.9317              & 0.9418 \\
\hline
Extended & 41.3801       & 0.9858            & 0.9372              & 0.9474 \\
\hline
\end{tabular}
\caption{Experiment results}
\end{table}
From the results in Table 1, the \textit{Extended} model performed better in all four aspects. We believe that this is because the $w_1$ parameter makes the model more flexible, so that it can cross some local optimal regions more easily. We also noticed that the \textit{Extended} model had lower training loss than the \textit{Original} model, while for other three aspects, especially for validation accuracy and test score, these two models had no big differences. We think this can be explained by the difference of model complexity. The \textit{Extended} model is more complex than the \textit{Original} model, so it can fit the training data better, but the accuracy is too high to improve much.

\section{Remark on Contributions}
The project is finished under the discussion and close collaboration of our group members. Jinxing Yu wrote the code skeleton and wrote the model part. Hao Xin and Xun Jian proposed different changes to the models. Hao Xin wrote the introduction part and Xun Jian wrote the discussion part.

% \subsubsection*{Acknowledgments}

% Use unnumbered third level headings for the acknowledgments. All
% acknowledgments go at the end of the paper. Do not include
% acknowledgments in the anonymized submission, only in the final paper.

% \section*{References}

% References follow the acknowledgments. Use unnumbered first-level
% heading for the references. Any choice of citation style is acceptable
% as long as you are consistent. It is permissible to reduce the font
% size to \verb+small+ (9 point) when listing the references. {\bf
%   Remember that you can use a ninth page as long as it contains
%   \emph{only} cited references.}
% \medskip

% \small

% [1] Alexander, J.A.\ \& Mozer, M.C.\ (1995) Template-based algorithms
% for connectionist rule extraction. In G.\ Tesauro, D.S.\ Touretzky and
% T.K.\ Leen (eds.), {\it Advances in Neural Information Processing
%   Systems 7}, pp.\ 609--616. Cambridge, MA: MIT Press.

% [2] Bower, J.M.\ \& Beeman, D.\ (1995) {\it The Book of GENESIS:
%   Exploring Realistic Neural Models with the GEneral NEural SImulation
%   System.}  New York: TELOS/Springer--Verlag.

% [3] Hasselmo, M.E., Schnell, E.\ \& Barkai, E.\ (1995) Dynamics of
% learning and recall at excitatory recurrent synapses and cholinergic
% modulation in rat hippocampal region CA3. {\it Journal of
%   Neuroscience} {\bf 15}(7):5249-5262.

\end{document}
